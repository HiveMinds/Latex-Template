% Glossary is a list of terms and their description.
% Source: https://www.overleaf.com/learn/latex/Glossaries
% All before begin document
%\usepackage[toc]{glossaries} % if no acronyms are used.
%\usepackage[acronym,toc]{glossaries} at acronyms.tex is sufficient.
%\makeglossaries
%\mbox{}
\newglossaryentry{latex}
{
        name=latex,
        description={Is a mark up language specially suited for
scientific documents}
}

\newglossaryentry{maths}
{
        name=mathematics,
        description={Mathematics is what mathematicians do}
}

\newglossaryentry{formula}
{
        name=formula,
        description={A mathematical expression}
}

\newglossaryentry{ptypesemiconductor}
{
        name=p-type semiconductor,
        description={A semiconductor that is doped with electron acceptors}
}

\newglossaryentry{ntypesemiconductor}
{
        name=n-type semiconductor,
        description={A semiconductor that is doped with electron donors}
}

\newglossaryentry{pnjunction}
{
        name=p-n junction,
        description={A boundary interface of two  a p-type semiconductor and a n-type semiconductor}
}

\newglossaryentry{holes}
{
        name=(electron) holes,
        description={In the context of doped semiconductors, holes are positions where electron acceptors are located, in other words, they are holes at which the electron can go.}
}

\newglossaryentry{unipolar-transistors}
{
        name=unipolar transistors,
        description={Unipolar transistors are transistors that use either electrons or electron holes as charge carriers and not the combination of the two.}
}
\newglossaryentry{cpu_cache}
{
        name=CPU cache,
        description={A small hardware memory unit that is faster than the main memory and closer to the \acrlong{alu}.}
}
\newglossaryentry{instruction_cache}
{
        name=instruction cache,
        description={A cache that is designed to increase the speed with which instructions are fetched.}
}
\newglossaryentry{data_cache}
{
        name=data cache,
        description={A cache that is designed to increase the speed with which data is fetched and stored.}
}
\newglossaryentry{random-access}
{
        name=random-access,
        description={A memory type that has access times that are independent of its physical location.}
}
\newglossaryentry{memory_cell}
{
        name=memory cell,
        description={A fundamental/basic unit in computing that is used to store information.}
}

\newglossaryentry{primary_memory}
{
        name=primary memory,
        description={A form of memory that is only accessible to the \acrlong{cpu} which reads instructions from it and executes those instructions.}
}

\newglossaryentry{prompt_charge}
{
        name=prompt charge,
        description={The charge that is collected by means of funnelling.} % TODO: verify whether this lose interpretation of:
        %A large fraction of the total charge collected by the circuit node occurs in time periods of about 200 ps, and this is referred to as prompt charge. There is also a delayed component that is collected by diffusion. The delayed component can extend to 1 µs or longer,[2] and is important for slower SEE phenomena such as upset in dynamic memories, and latchup.
        % from:
        %https://radhome.gsfc.nasa.gov/radhome/papers/seeca4.htm
        % is accurate.
}
%\printglossary
